%!TEX root = ../ausarbeitung.tex
\section{Fazit}
\label{sec:fazit}

In dieser Ausarbeitung haben wir zahlreiche Methoden gesehen, mit deren Hilfe sich alle Probleme, die bei der Bearbeitung eines Projekts mit Kamerafallenbildern anfallen, lösen lassen. Der Camera Trap Sequencer sortiert einzelne Ordner oder ganze Datenbanken in zusammenhängende Sequenzen. Er ist über ein unkompliziertes grafisches Tool benutzbar.

Die Sequenzen können anschließend mithilfe des PCA-Verfahrens über die Berechnung eines Hintergrundbilds pro Sequenz segmentiert werden. Dadurch lassen sich relevante Bildausschnitte bestimmen und Tiere lokalisieren. Dank der ausgefeilten Pipeline aus Vor- und Nachbereitungsschritten sind überzeugende Segmentierungsergebnisse möglich. Gerade die Identifizierung von mehreren ROIs im gleichen Bild ist bei Bildern von Tiergruppen oder -herden ein unschätzbarer Vorteil. Sollten Bilder einmal nicht in Sequenzen, sondern als einzelne Dateien vorliegen, können wir das Sliding-Window-Verfahren anwenden, welches Lokalisierung und Klassifizierung miteinander verbindet.

In der ersten Iteration des Projekts haben wir einen SVM-Klassifikator implementiert, der einen RBF-Kernel auf HOG-Features benutzt. Das erste Ziel einer Präzision von über 80~\% auf der DDD wurde mit durchschnittlich 83~\% auf dem besten Parametersatz erreicht. Zudem zeichnet sich der Algorithmus durch seine schnelle Laufzeit aus. Leider sinkt die Güte der Ergebnisse bei komplizierteren Klassifizierungsproblemen deutlich ab, sodass ein Einsatz auf der DDD+ nicht sinnvoll war.

Für die Klassifizierung eines komplizierten Datensatzes mit vielen Klassen und Tag- und Nachtbildern, haben wir Spatial Pyramid Matching mit LLC implementiert. Dabei werden auf immer feineren Teilbildern SIFT- oder LBP-Deskriptoren berechnet und auf einem Codebook lokal encodiert. Die LLC-Codes lassen sich anschließend mit einer linearen SVM klassifizieren. Für dieses Verfahren werden auf der kleinen DDD Ergebnisse erreicht, die der von Menschen ebenbürtig ist. Auch auf der komplizierteren DDD+ lag der F1-Score fast bei 80~\%. Durch die Implementierung als \texttt{BaseEstimator} lässt sich SPM leicht in Scikit-learn Pipelines integrieren, was eine Vielzahl von Ensemblemethoden und Modellselektionsverfahren erlaubt.

Das größte Problem unserer Verfahren ist momentan die schlechte Laufzeit des Spatial Pyramid Matchings. Während das Training der SVM sehr effizient möglich ist und das trotz Codes, die mehrere zehntausend Elemente lang sind, verbraucht diese Codierung sehr viel Zeit. Das nächste Ziel ist deshalb eine Reimplementierung des LLC-Verfahrens mit Tensorflow, um eine höhere Parallelität zu erreichen. Durch die Verwendung von größeren Codebooks und der besseren Möglichkeit die optimalen Hyperparameter zu suchen, sind dadurch auch bessere Klassifikationsergebnisse denkbar. \\
Auch komplexere Ensemblemethoden, wie beispielsweise \emph{AdaBoost} oder \emph{VotingClassifier}, sind Ansätze, die unser Ergebnis verbessern könnten \cite{ywkjwh13}. Des Weiteren wäre es sinnvoll die Sliding-Window-Methoden so zu erweitern, dass man einen eigenen vortrainierten Klassifikator benutzen kann, der dann für die Lokalisierung und Klassifizierung benutzt wird.