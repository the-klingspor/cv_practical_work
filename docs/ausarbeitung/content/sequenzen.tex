%!TEX root = ../ausarbeitung.tex
\section{Aufteilung auf Sequenzen}

\subsection{Sortierung mithilfe der EXIF-Daten}

Die handelsüblichen Kamerafallen haben einen Sensor, der bei Bewegung auslöst und zunächst zehn Bilder im Abstand von jeweils einer Sekunde schießt. Sollte es am Ende einer Zehnersequenz weiterhin Bewegung im Sichtfeld der Kamera geben, löst der Sensor erneut aus und es werden weitere zehn Bilder aufgenommen. Das sorgt dafür, dass der Datensatz aus zusammenhängenden Sequenzen von jeweils zehn oder mehr Bildern besteht. Oft befinden sich gerade auf den letzten Aufnahmen einer Sequenz keine Tiere mehr, da sich diese aus dem Bild bewegt haben. 

Unglücklicherweise sind die Daten auf der Datenbank lediglich nach Tierart sowie dort jeweils nach Tag und Nacht, leeren Bildern und Fehlklassifizierungen sortiert. Für das korrekte Funktionieren unserer Segmentierungstechnik PCA ist es aber nötig, dass die Daten in zusammenhängenden Sequenzen vorliegen. Aus diesem Grund benutzen wir die EXIF-Metadaten, um  
-EXIF
-EXIF Tool

\subsection{Camera Trap Sequencer}
-Optionen
-Beschreibung des Aufbaus des Datensatzes
-Implementierung mit qt5