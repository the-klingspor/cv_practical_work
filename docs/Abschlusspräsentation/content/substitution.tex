%!TEX root = ../vortrag.tex
\section{Das Substitutionsmodell}
\begin{frame}[t,fragile]{Prozeduren}
	\begin{itemize}
		\item Prozeduren sind wesentlich zur Abstraktion und Modularisierung von Berechnungen \pause
		\item Erzeugung einer Prozedur in Racket / Scheme mittels \mintinline{scheme}{lambda}:
		\begin{center}
			\texttt{(lambda (\slot{Parameterliste}) \slot{Rumpf})}
		\end{center} \pause
		\item \mintinline{scheme}{lambda} liefert eine Prozedur als Rückgabewert, die Bestandteil zusammengesetzter Ausdrücke sein kann:
		\begin{minted}[fontsize=\normalsize]{scheme}
			> (lambda (x) (* x x))	;; quadriert einen übergebenen Wert
			#<procedure>
			
			> ((lambda (x) (* x x)) 6)
			36
		\end{minted}
		\pause
		\item Wiederverwendbarkeit durch \bet{Bindung} der Funktion an einen Namen (siehe nächste Folie).
	\end{itemize}
\end{frame}

\begin{frame}[t,fragile]{Bindungen}
	\begin{itemize}
		\item Bindung von Ausdrücken an Namen mittels \mintinline{scheme}{define}:
		\begin{center}
			\texttt{(define \slot{Name} \slot{Ausdruck})}
		\end{center} \pause
		\begin{minted}{scheme}
			(define square (lambda (x) (* x x)))
			(define sqrt2 1.4142)
			
			> (square 6)
			36
			> (square sqrt2)
			1.9999616399999998
		\end{minted}
		\pause
		\item Kurzschreibweise zur Definition und Bindung von Prozeduren:
		\begin{center}
			\texttt{(define (\slot{Name} \slot{Parameterliste}) \slot{Rumpf})}
		\end{center}
		\begin{minted}{scheme}
			(define (square x) (* x x))
			
			> (square 6)
			36
		\end{minted}
	\end{itemize}
\end{frame}

\begin{frame}[t,fragile]{Verschachteln von Prozeduren}
	\begin{itemize}
		\item Prozeduren, die an einen Namen gebunden sind, können in anderen Prozeduren verwendet werden:
		\begin{minted}[fontsize=\normalsize]{scheme}
			(define (sum-square x y)		;; x^2 + y^2
			    (+ (square x) (square y)))
			
			(define (distance x y)			;; Abstand von (x,y) zum Ursprung
			    (sqrt (sum-square x y)))
			
			> (sum-square 3 4)
			25
			
			> (distance 3 4)
			5
		\end{minted}
	\end{itemize}
\end{frame}

\begin{frame}[t,fragile]{Prozeduren als Datentyp}
	\begin{itemize}
		\item Prozeduren können als Parameter an andere Prozeduren übergeben werden:
		\begin{minted}[fontsize=\normalsize]{scheme}
		(define (vier-drei f)
			(f 4 3))
			
		> (vier-drei +)
		7
		> (vier-drei *)
		12
		> (vier-drei sum-square)
		25
		> (vier-drei (lambda (x y) (- (* x x x) (* x y))))
		52
		\end{minted}
	\end{itemize}
\end{frame}

\begin{frame}[t,fragile]{}
	\begin{itemize}
		\item Prozeduren können das Ergebnis einer anderen Prozedur sein:
		\only<1|handout:0>{\inputminted[firstline=5,lastline=11]{scheme}{code/return-fun.scm}}
		\only<2->{\inputminted[firstline=5]{scheme}{code/return-fun.scm}}
	\end{itemize}
\end{frame}

\begin{frame}[t,fragile]{Auswertung von Prozeduren}
	\begin{itemize}
		\item Prozeduren sind in Scheme bzw. Racket damit \bet{Daten erster Klasse} \ccite{sicp}{76}, denn sie können
		\begin{itemize}
			\item an Namen gebunden werden,
			\item als Parameter an Prozeduren übergeben werden,
			\item als Rückgabewert einer Prozedur auftreten.
			\minoritem \minor{Bestandteil einer Datenstruktur sein.}
		\end{itemize} \pause
		\item Wie kann die Auswertung einer Prozedur nachvollzogen werden?
		\begin{mybox}
			Auswertung eines zusammengesetzten Ausdrucks $\mathtt{(op \ exp_1 \ exp_2 \ \dots \ exp_k)}$:
			\begin{enumerate}[(1)]
				\item Werte die Teilausdrücke $\mathtt{op}, \mathtt{exp_1}, \mathtt{exp_2}, \dots, \mathtt{exp_k}$ aus.
				\item Wende die Auswertung von $\mathtt{op}$ auf die Auswertungen von $\mathtt{exp_1}, \dots, \mathtt{exp_k}$ an.
			\end{enumerate}
		\end{mybox} \pause
		\item[$\Rightarrow$] Ergänzung des Auswertungsmodells zur Auswertung nicht-elementarer Prozeduren
	\end{itemize}
\end{frame}

\begin{frame}[t,fragile]{Substitutionsmodell}
	\begin{mybox}
		Um eine Prozedur auf konkrete Werte anzuwenden, ersetze im Rumpf der Prozedur jeden formalen Parameter mit dem entsprechenden Wert und werte den Rumpf aus. \ccite{sicp}{14}
	\end{mybox} \pause
	
	\vspace*{0.5cm}
	
	\bet{Beispiel 1:} Auswertung von \mintinline{scheme}{(distance (+ 4 1) (* 3 4))}
	\begin{minted}{scheme}
	(define (square x) (* x x))
	(define (sum-square x y) (+ (square x) (square y)))
	(define (distance x y) (sqrt (sum-square x y)))
	\end{minted}
\end{frame}

\begin{frame}[t,fragile]{}
	\begin{mybox}
		Um eine Prozedur auf konkrete Werte anzuwenden, ersetze im Rumpf der Prozedur jeden formalen Parameter mit dem entsprechenden Wert und werte den Rumpf aus.
	\end{mybox}
	\begin{minted}{scheme}
	(define (square x) (* x x))
	(define (sum-square x y) (+ (square x) (square y)))
	(define (distance x y) (sqrt (sum-square x y)))
	\end{minted}
	
	\vspace*{0.5cm}
	
	\begin{tabular}{l}
		\mintinline{scheme}{(distance (+ 4 1) (* 3 4))} \\[0.2cm]
		\onslide<2->{$\rightarrow$ \mintinline{scheme}{(sqrt (sum-square 5 12))}} \\[0.2cm]
		\onslide<3->{$\rightarrow$ \mintinline{scheme}{(sqrt (+ (square 5) (square 12)))}} \\[0.2cm]
		\onslide<4->{$\rightarrow$ \mintinline{scheme}{(sqrt (+ (* 5 5) (* 12 12)))}} \\[0.2cm]
		\onslide<5->{$\rightarrow$ \mintinline{scheme}{(sqrt (+ 25 144))}} \\[0.2cm]
		\onslide<6->{$\rightarrow$ \mintinline{scheme}{(sqrt 169)}} \\[0.2cm]
		\onslide<7->{$\rightarrow$ \mintinline{scheme}{13}}
	\end{tabular}
\end{frame}

\begin{frame}[t,fragile]{} \label{folie:add-fun-eval}
	\bet{Beispiel 2:} Auswertung von \mintinline{scheme}{((add-fun square cube) 5)}
	\begin{minted}{scheme}
	(define (square x) (* x x))
	(define (cube x) (* x x x))
	(define (add-fun f g)
		(lambda (x) (+ (f x) (g x))))
	\end{minted}
	
	\vspace*{0.5cm}
	
	\begin{tabular}{l}
		\mintinline{scheme}{((add-fun square cube) 5)} \\[0.2cm]
		\onslide<2->{$\rightarrow$ \mintinline{scheme}{((lambda (x) (+ (square x) (cube x))) 5)}} \\[0.2cm]
		\onslide<3->{$\rightarrow$ \mintinline{scheme}{(+ (square 5) (cube 5))}} \\[0.2cm]
		\onslide<4->{$\rightarrow$ \mintinline{scheme}{(+ (* 5 5) (* 5 5 5))}} \\[0.2cm]
		\onslide<5->{$\rightarrow$ \mintinline{scheme}{(+ 25 125)}} \\[0.2cm]
		\onslide<6->{$\rightarrow$ \mintinline{scheme}{150}}
	\end{tabular}
\end{frame}

\begin{frame}[t,fragile]{Auswertungsprozess im Racket-Stepper}
	\begin{itemize}
		\item Sprachstufe \textit{Intermediate Student with lambda}
	\end{itemize}
	\begin{center}
		\includegraphics[keepaspectratio,width=14cm]{img/stepper.png}
	\end{center}
\end{frame}

\begin{frame}[t,fragile]{Substitutionsmodell}
	\begin{mybox}
		Um eine Prozedur auf konkrete Werte anzuwenden, ersetze im Rumpf der Prozedur jeden formalen Parameter mit dem entsprechenden Wert und werte den Rumpf aus.
	\end{mybox}
	
	\vspace*{0.5cm}
	
	\bet{Bemerkungen:} \pause
	\begin{itemize}
		\item \bet{Wichtig:} Das Substitutionsmodell hat den Zweck, die Auswertung eines Prozeduraufrufs nachvollziehbar zu machen.
		Es gibt jedoch \bet{nicht} an, wie sich ein Interpreter verhalten muss. \pause
		\item Wann genau werden elementare Prozeduren ausgewertet?
		\begin{itemize}
			\item Sobald möglich oder erst, wenn nötig?
		\end{itemize}
	\end{itemize}
\end{frame}

\begin{frame}[fragile]{}
	\begin{minipage}{0.45\textwidth}
		\mintinline[fontsize=\small]{scheme}{(sum-square (+ 4 1) (* 3 4))} \\[0.25cm]
		\mintinline[fontsize=\small]{scheme}{(+ (square 5) (square 12))} \\[0.25cm]
		\mintinline[fontsize=\small]{scheme}{(+ (* 5 5) (* 12 12))} \\[0.25cm]
		\mintinline[fontsize=\small]{scheme}{(+ 25 144)} \\[0.25cm]
		\mintinline[fontsize=\small]{scheme}{169} \\[0.7cm]
		\begin{center}
			\textit{applicative order}
			
			\textit{evaluation}
		\end{center}
	\end{minipage}~\begin{minipage}{0.55\textwidth}
		\mintinline[fontsize=\small]{scheme}{(sum-square (+ 4 1) (* 3 4))} \\[0.25cm]
		\mintinline[fontsize=\small]{scheme}{(+ (square (+ 4 1)) (square (* 3 4))} \\[0.25cm]
		\mintinline[fontsize=\small]{scheme}{(+ (* (+ 4 1) (+ 4 1)) (* (* 3 4) (* 3 4)))} \\[0.25cm]
		\mintinline[fontsize=\small]{scheme}{(+ (* 5 5) (* 12 12))} \\[0.25cm]
		\mintinline[fontsize=\small]{scheme}{(+ 25 144)} \\[0.25cm]
		\mintinline[fontsize=\small]{scheme}{169} \\[0cm]
		\begin{center}
			\textit{normal order}
			
			\textit{evaluation}
		\end{center}
	\end{minipage}
\end{frame}

\begin{frame}[t,fragile]{Substitutionsmodell}
	\begin{mybox}
		Um eine Prozedur auf konkrete Werte anzuwenden, ersetze im Rumpf der Prozedur jeden formalen Parameter mit dem entsprechenden Wert und werte den Rumpf aus.
	\end{mybox}
	
	\vspace*{0.5cm}
	
	\bet{Bemerkungen:}
	\begin{itemize}
		\item \bet{Wichtig:} Das Substitutionsmodell hat den Zweck, die Auswertung eines Prozeduraufrufs nachvollziehbar zu machen.
		Es gibt jedoch \bet{nicht} an, wie sich ein Interpreter verhalten muss.
		\item Wann genau werden elementare Prozeduren ausgewertet? \label{folie:subst-bem}
		\begin{itemize}
			\item Sobald möglich oder erst, wenn nötig?
			\item \enquote{Arguments to Scheme procedures are always passed by value, which means that the actual argument expressions are evaluated before the procedure gains control} \ccite{r5rs}{Kap. 1.1} \pause
			\item Darüber hinaus gibt es so genannte \bet{special forms} mit eigenen Auswertungsregeln: \mintinline{scheme}{define}, \mintinline{scheme}{lambda}, \mintinline{scheme}{if}, \mintinline{scheme}{cond}, \dots
		\end{itemize}		
	\end{itemize}
\end{frame}