%!TEX root = ../vortrag.tex
\section{Evaluierung}
\begin{frame}[t,fragile]{DDD nur Tagbilder}
\begin{itemize}
	\item 70\% aller Bilder zum Trainieren, 30\% zum Testen
\end{itemize}
	\begin{table}
		\begin{tabular}{|c||c|c|c|c|}
			\hline
			Mean Accuracy & $\lambda$ = 500 & $\lambda$ = 100 & $\lambda$ = 10 & $\lambda$ = 1 \\
			\hline
			$\sigma$ = 100 & 94,3\% & 90.5\% & 93,3\% & 87,6\% \\ 
			\hline
			$\sigma$ = 10 & 87,6\% & - & - & 87,6\% \\
			\hline
		\end{tabular}
	\caption{Größe des Codebooks: 256 Features}
	\end{table}
		\begin{table}
		\begin{tabular}{|c||c|c|c|c|}
			\hline
			Mean Accuracy & $\lambda$ = 500 & $\lambda$ = 100 & $\lambda$ = 10 & $\lambda$ = 1 \\
			\hline
			$\sigma$ = 100 & 93,3\% & 95,7\% & 87,6\% & 87,6\% \\ 
			\hline
			$\sigma$ = 10 & 87,6\% & - & - & 87,6\% \\
			\hline
		\end{tabular}
		\caption{Größe des Codebooks: 512 Features}
	\end{table}
\end{frame}

\begin{frame}[t, fragile]{DDD+}
	\begin{itemize}
		\item größere Testdatenbank mit Tag- und Nachtbildern von Damhirsch, Feldhase, Dachs, Wildschaf, Wildschwein und Rotfuchs
		\item insgesamt 2336 Bilder, zwischen 240 und 490 pro Tierart
		\item Confusion Matrix:
		\item Mean Accuracy = 
	\end{itemize}
\end{frame}
